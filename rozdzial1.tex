
\chapter{Historia początków samochodów}

\section{Pierwszy silnik}

Początki motoryzacji sięgają XVIII wieku, 
gdy francuski wynalazca Nicolas-Joseph Cugnot w 1769 roku 
zbudował jeden z pierwszych pojazdów który był samojezdny
oraz napędzany parą. Jednakże stworzona przez niego maszyna
była zbyt ciężka i powolna aby zmienić trajektorię historii.
Prawdziwa rewolucja nastąpiła w XIX wieku kiedy to 
Etienne Lenoir zbudował pierwszy użyteczny, dwusuwowy silnik
spalinowy który napędzał pojazd. 

Parę lat później bo w 1876 Nikolausa Otto wynalazł 
pierwszy praktyczny, czterosuwowy silnik spalinowy, 
który stał się wzorem dla współczesnych silników. 
Sama koncepcja silnika spalinowego została zaproponowana 
przez Philippe Lebon w 1799 roku którą idee później 
oparto się przy konstrukcji pierwszego praktycznego silnika.

\section{Pierwszy samochód}

Na skraju XIX wieku tak o to Karl Benz i Gottlieb Daimler, wraz ze swoimi
wynalazkami zapoczątkowali erę samochodów spalinowych. W tym okresie
powstały również pierwsze funkcjonalne pojazdy parowe, które
zachowały się do dzisiaj. Obecnie za najstarszy samochód
uznawany jest De Dion-Bouton et Trépardoux Dos-à-Dos Steam Runabout
z 1884 roku, wyprodukowany przez francuską firmę De dion, Bouton 
Trépardoux, był napędzany parą wodną. Posiadał zbiornik na wodę,
kotły parowe i prosty układ kierowniczy. Zasięg wynosił około 32 km
a prędkość maksymalna około, 60 km/h.

Prawdziwym przełom nastąpił dzieki Karlowi Benzowi, kiedy to w
1885 roku stworzył pierwszy samochód napędzany silnikiem benzynowym
o nazwie Benz Patent-Motorwagen
Motorwagen pozostał fundamentem nowoczesnej motoryzacji, był pojazdem
trójkołowym wyposażonym w jednocylindrowy silnik spalinowy.


\section{Jak działa samochód - zasady działania i nowoczesne technologie}

Samochód to jedno z najbardziej fascynujących osiągnięć ludzkiej
techniki motoryzacyjnej. Od momentu wynalezienia pierwszych pojazdów
napędzanych silnikiem spalinowym przez Karla Benza w XIX wieku,
motoryzacja przeszła ogromną ewolucję. Współczesne samochody są połączeniem
mechaniki, elektroniki i informatyki jako wspólnego systemu współpracującego ze
sobą, tworząc złożoną maszynę. Aby zrozumieć, jak działa
samochód, należy przyjrzeć się zarówno podstawowym zasadom jego
działania, jak i nowoczesnym rozwiązaniom oraz technologiom, które bieżąco zmieniają
motoryzację.

\begin{figure}[H]
	\centering
	\includegraphics[width=0.7\textwidth]{obrazek2}
	\caption{Pierwszy samochód, Źródło: https://cuk.pl}
\end{figure}

Przedstawiony na fotografii Patentwagen numer 1 był produkowanego od 1886 do 1893,
jest pierwszym samochodem z silnikiem spalinowym
wyprodukowanego przez Carla Friedricha Benza. Pojazd osiągał prędkość
maksymalną ok. 16 km/h przy mocy około 1 KM.
Obecnie samochody dzięki nowym technikom wyewoluowały w o wiele szybsze
i posiadające mocniejsze silniki maszyny zdolne do przekraczania
zawrotnych prędkości. W 1998 roku tytuł najszybszego fabrycznie
produkowanego samochodu otrzymał McLaren F1, który zdołał rozpędzić się do
386,7 km/h. Do dziś trzyma on rekord najszybszego auta z silnikiem
wolnossącym.

\section{Podstawowa zasada działania samochodu}

Na najbardziej ogólnym poziomie, samochód jest urządzeniem, które
przekształca energię chemiczną paliwa w energię mechaniczną potrzebną do
poruszania się. W tradycyjnych pojazdach z silnikiem spalinowym proces
ten zachodzi w silniku, który spala paliwo w kontrolowany sposób.
Energia cieplna wytwarzana podczas spalania przekształca się w ruch
tłoków, a następnie za pomocą korbowodu i wału korbowego  w ruch
obrotowy. Ten ruch przekazywany jest przez skrzynię biegów i układ
przeniesienia napędu do kół, co powoduje, że pojazd zaczyna się
poruszać.

\section{Serce samochodu - silnik spalinowy}

Silnik spalinowy to serce tradycyjnego samochodu. Najczęściej spotykane
są silniki czterosuwowe, które składają się z czterech etapów pracy:
ssania, sprężania, spalania i wydechu. Podczas suwu ssania do cylindra
trafia mieszanka paliwa i powietrza, która następnie zostaje sprężona
przez tłok. W kolejnym etapie mieszanka zostaje zapalona przez iskrę ze
świecy zapłonowej, co powoduje gwałtowne rozszerzenie gazów i ruch tłoka
w dół. W ostatnim etapie spaliny są usuwane z cylindra przez zawory
wydechowe.

\begin{figure}[H]
	\centering
	\includegraphics[width=0.7\textwidth]{obrazek3}
	\caption{Silnik spalinowy, Źródło: https://hondagrudzinski.pl}
\end{figure}
