\chapter{Podstawy działania samochodu}

We współczesnych samochodach rozróżniamy głównie cztery rodzaje silników

\begin{enumerate}
	\item Silnik Benzynowy
	\begin{enumerate}
		\item Jest zasilany przez benzyne, posiada wysokie obroty, oraz
			jest płynny w pracy. Tańszy w produkcji niż Diesel
			oraz tańsze koszty naprawy. Jednak zużywa więcej paliwa
			i generuje większą emisje spalin.
	\end{enumerate}
	\item Silnik Diesel
	\begin{enumerate}
		\item Zasilany przez olej napędowy, mniejsze spalanie niż w silnikach
			benzynowych, elastyczny, idealny do dłuższych tras ale 
			zwiększony koszt przy ewentualnych usterkach. 
			Jest mniej trwały niż silniki 
			benzynowe.
	\end{enumerate}

	\item Silnik Hybrydowy
	\begin{enumerate}
		\item Zasilany przez prąd elektryczny i benzyne
			Łączy zalety obu silników przez co
			zapewnia niższe zużycie paliwa i mniejszą
			emisje spalin, jednakże w wielu modelach
			ma mniejszy bagażnik.
	\end{enumerate}
	\item Silnik elektryczny
	\begin{enumerate}
		\item Zasilany przez prąd elektryczny, posiada 
			zerową emisje spalin, jest cichy w pracy
			Wadą jest ograniczony zasięg oraz długi czas
			ładowanie i wysokie koszty.
	\end{enumerate}
\end{enumerate}


Nowoczesne silniki są znacznie bardziej zaawansowane. Zastosowanie
turbosprężarek, systemów wtrysku bezpośredniego, komputerowego
sterowania zapłonem i czujników tlenu pozwala na zwiększenie mocy przy
jednoczesnym zmniejszeniu zużycia paliwa i emisji spalin. Coraz częściej
spotyka się również systemy start i stop, które automatycznie wyłączają
silnik podczas postoju, co dodatkowo oszczędza paliwo.

\section{Rodzaje silników spalinowych - silnik wolnossący} 

Silnik wolnossący to konstrukcja, w której powietrze zasysane do
cylindrów trafia tam wyłącznie dzięki naturalnemu podciśnieniu
powstającemu podczas suwu ssania. Nie wykorzystuje on urządzeń
wspomagających, takich jak turbosprężarki czy kompresory. Dzięki temu
jego budowa jest prostsza, a reakcja na wciśnięcie pedału gazu jest
bardziej przewidywalna i płynna.

Silniki wolnossące cechują się liniowym oddawaniem mocy i wysoką
trwałością, ponieważ nie są narażone na tak duże obciążenia termiczne i
mechaniczne jak jednostki doładowane. Wadą jest niższy moment obrotowy
przy niskich obrotach oraz mniejsza moc w porównaniu z jednostkami
turbodoładowanymi o tej samej pojemności.

\section{Układ napędowy i przeniesienie mocy w samochodach}

Sam silnik nie wystarczy, aby samochód mógł się poruszać. Potrzebny jest
układ, który przekaże moment obrotowy z wału korbowego na koła. Kluczową
rolę odgrywa tu skrzynia biegów, która umożliwia zmianę przełożenia i
dostosowanie obrotów silnika do prędkości jazdy. W samochodach
manualnych kierowca sam wybiera biegi, natomiast w automatycznych proces
ten odbywa się automatycznie dzięki systemowi hydraulicznemu lub
elektronicznemu.

Następnie moment obrotowy trafia do mechanizmu różnicowego, który
rozdziela moc między koła, pozwalając im obracać się z różną prędkością
podczas skrętu. W zależności od konstrukcji samochodu, napęd może być
przenoszony na przednie koła (FWD), tylne (RWD) lub wszystkie (AWD).


\section{Układ kierowniczy i zawieszenie}

Układ kierowniczy umożliwia kierowcy zmianę kierunku jazdy. W większości
samochodów stosuje się przekładnię zębatkową, która przekształca ruch
obrotowy kierownicy w ruch kół. Współczesne samochody
wyposażone są we wspomaganie kierownicy np. hydrauliczne, elektryczne lub
elektrohydrauliczne co znacznie ułatwia manewrowanie i ogólne poruszanie się
samochodem.

Zawieszenie pełni rolę pośrednika między kołami a nadwoziem. Jego
zadaniem jest tłumienie drgań i zapewnienie przyczepności kół do
podłoża. W skład zawieszenia wchodzą amortyzatory, sprężyny, wahacze i
stabilizatory. Nowoczesne samochody coraz częściej wyposażone są w
adaptacyjne zawieszenie, które automatycznie dostosowuje twardość
amortyzatorów do warunków na drodze.

\section{Układ hamulcowy}

Bez skutecznych hamulców żaden samochód nie mógłby być bezpieczny.
Współczesne samochody korzystają z hamulców tarczowych, które działają
dzięki zaciskom ściskającym tarczę obracającą się wraz z kołem. 

W układzie hamulcowym istotną rolę odgrywa płyn hamulcowy, który przenosi
ciśnienie z pedału hamulca na zaciski. Nowoczesne systemy wspomagające,
takie jak ABS czy ESP, zwiększają bezpieczeństwo, zapobiegając
blokowaniu się kół i utracie kontroli nad pojazdem.

\section{Elektronika i komputery w samochodzie}

Współczesne samochody to w dużej mierze komputery na kołach. Każdy
nowoczesny pojazd ma nawet kilkadziesiąt mikrokontrolerów, które
zarządzają pracą silnika, hamulców, klimatyzacji, systemów
bezpieczeństwa i multimediów. Kluczową rolę odgrywa ECU komputer
sterujący silnikiem, który analizuje dane z czujników i na ich podstawie
decyduje o ilości paliwa, kącie zapłonu czy ciśnieniu doładowania.

Poza tym samochody są wyposażone w liczne systemy wspomagania kierowcy,
takie jak automatyczne hamowanie awaryjne, utrzymanie pasa ruchu,
czy chociaż rozpoznawanie znaków drogowych. Wraz z rozwojem
technologii coraz bliżej jesteśmy aut które jeżdzą za nas.

W skrócie współczesne samochody posiadają:

\begin{itemize}
	\item Silnik
	\begin{itemize}
		\item Spalinowy lub elektryczny generujący moc.
	\end{itemize}
	\item Nadwozie
	\begin{itemize}
		\item Główną konstrukcję samochodu w tym przestrzeń pasażerską i bagażową
	\end{itemize}
	\item Układ napędowy
	\begin{itemize}
		\item Przekażujący moment obrotowy z wału korbowego na koła pojazdu
	\end{itemize}
	\item Układ kierowniczy i zawieszenia
	\begin{itemize}
		\item Umożliwiający zmianę kierunku jazdy
	\end{itemize}
	\item Układ hamulcowy
	\begin{itemize}
		\item Dla bezpieczeństwa
	\end{itemize}
	\item Elektronike i komputery
	\begin{itemize}
		\item Rózne wspomagania, systemy oświetlenia i podzespoły
	\end{itemize}
\end{itemize}

