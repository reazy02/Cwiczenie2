\section{Wnioski}

\begin{equation}
	E = mc^2
\end{equation}


\begin{tabular}{ l | c | r }

komórka 1 & komórka 2 & komórka 3 \\

\hline

komórka 4 & komórka 5 & komórka 6 \\

\end{tabular} 


\chapter{Przyszłość technologii w samochodach}

\section{Nowoczesne napędy oraz ekologia w motoryzacji}

Ostatnie lata przyniosły prawdziwą rewolucję w motoryzacji. Coraz
większy nacisk kładzie się na ochronę środowiska i redukcję emisji CO2,
co doprowadziło do rozwoju pojazdów elektrycznych i hybrydowych. W
samochodach elektrycznych tradycyjny silnik spalinowy zastąpiono
silnikiem elektrycznym, który pobiera energię z akumulatorów
litowo-jonowych. Energia ta napędza koła bezpośrednio, dzięki czemu
układ napędowy jest prostszy i bardziej efektywny.

\includegraphics{obrazek4}

   Fot. 4. Tesla Model S

Zaletą samochodów elektrycznych jest ich wysoka wyjdaność, cicha praca
i brak lokalnej emisji spalin. Wadą natomiast jest ograniczony zasięg i długi
czas ładowania, choć postęp w technologii baterii stopniowo rozwiązuje
te problemy. Hybrydy jednak są nadal rozwiązaniem pośrednim łączą silnik
spalinowy z elektrycznym, co pozwala na zmniejszenie zużycia paliwa i emisji.

\section{Przyszłość samochodów}

Motoryzacja stoi obecnie na progu nowej ery. Wraz z rozwojem sztucznej
inteligencji i komunikacji między pojazdami coraz bliższa staje się
wizja samochodów które będą mogły poruszać się bez
udziału kierowcy. Tego typu systemy analizują dane z kamer i radarów 
tworząc trójwymiarowy obraz otoczenia i podejmując decyzje w
czasie rzeczywistym.

Równocześnie rozwijane są co raz to nowecześniejsze systemy zarządzania ruchem oraz
infrastruktura drogowa która współpracująca z pojazdami. Dzięki temu przyszłe
samochody będą nie tylko bezpieczniejsze, ale również bardziej efektywne
energetycznie i przyjazne środowisku.


\section{Największe rekordy prędkości samochodó}

Poniższa tabelka przedstawia rekordy prędkości osiągnięte przez najszybsze auta produkcyjne na przełomie XX i XXI wieku:

\begin{table}[]
\begin{tabular}{cllll}
Samochód                                & Rok rekordu               & Prędkość maksymalna           & Moc silnika                  & Waga    \\
Lamborghini Diablo                      & 1990                      & 326 km/h                      & 492 KM                       & 1385 kg \\
McLaren F1                              & 1994                      & 387 km/h                      & 627 KM                       & 1138 kg \\ \cline{2-2} \cline{4-4}
\multicolumn{1}{c|}{Koenigsegg CCXR}    & \multicolumn{1}{l|}{2007} & \multicolumn{1}{l|}{400 km/h} & \multicolumn{1}{l|}{1018 KM} & 1180 kg \\ \cline{2-2} \cline{4-4}
\multicolumn{1}{l}{Hennessey Venom GT}  & 2014                      & 435 km/h                      & 1244 KM                      & 1244 kg \\
\multicolumn{1}{l}{Koenigsegg Agera RS} & 2017                      & 447 km/h                      & 1341 KM                      & 1295 kg \\
\multicolumn{1}{l}{Bugatti Chiron SS}   & 2019                      & 490 km/h                      & 1600 KM                      & 1978 kg
\end{tabular}
\end{table}


\section{Podsumowanie}

Samochód to niezwykle złożony system, w którym współpracują mechanika,
elektronika i informatyka. Od prostych konstrukcji sprzed stu lat po
dzisiejsze hybrydy i auta elektryczne, motoryzacja przeszła długą drogę
i ciągle się rozwija stając się symbolem postępu technologicznego. Odkrywanie tego, 
jak działa samochód oraz jego podzespołów pozwala nam zrozumieć 
jak inżynierowie łączą naukę, technikę i innowacje, aby
tworzyć coraz doskonalsze bijące kolejne rekordy maszyny.



