\chapter{Przyszłość technologii w samochodach}

\section{Nowoczesne napędy oraz ekologia w motoryzacji}

Ostatnie lata przyniosły prawdziwą rewolucję w motoryzacji. Coraz
większy nacisk kładzie się na ochronę środowiska i redukcję emisji CO2,
co doprowadziło do rozwoju pojazdów elektrycznych i hybrydowych. W
samochodach elektrycznych tradycyjny silnik spalinowy zastąpiono
silnikiem elektrycznym, który pobiera energię z akumulatorów
litowo-jonowych. Energia ta napędza koła bezpośrednio, dzięki czemu
układ napędowy jest prostszy i bardziej efektywny.

\begin{figure}[H]
	\centering
	\includegraphics[width=0.7\textwidth]{obrazek4}
	\caption{Tesla Model S, Źródło: https://www.caranddriver.com/tesla/model-s}
\end{figure}

Istnieją jednak rodzaje samochodów hybrydowych z których napopularniejsze są trzy;
MHEV (Mild hybrid Eletric Vehicle) - najczęściej określane jako "miękka hybryda"
Jest to rozwiązanie pośrednie, wspierające silnik spalinowy za pomocą
niewielkiego motoru elektrycznego. System nie pozwala na jazdę wyłącznie na prąd
ale odciąża silnik przy ruszaniu i przyspieszaniu.

Kolejnym typem jest HEV (Hybrid Electric Vehicle) - w odróżnieniu od MHEV
potrafi on przez krótki czas jechać w trybie elektrycznym. Energia jest
odzyskiwana w trakcie hamowania lub pracy silnika spalinowego.

PHEV (Plug-in Hybrid Electric Vehicle) - posiada pojemniejszy akumulator
niż klasyczna hybryda to tego stopnia że można nim przejechać kilkadziesiąt
kilometrów wyłącznie na silniku elektrycznym. Gdy wyczerpnie się energia z
baterii, auto działa jak zwykła hybryda. Umożliwia ładowanie z gniazdka tak
jak auto w pełni elektryczne.

Zaletą samochodów elektrycznych jest ich wysoka wyjdaność, cicha praca
i brak lokalnej emisji spalin. Wadą natomiast jest ograniczony zasięg i długi
czas ładowania, choć postęp w technologii baterii stopniowo rozwiązuje
te problemy. Samochody Hybrydywe są jednak nadal rozwiązaniem pośrednim łączą silnik
spalinowy z elektrycznym, co pozwala na zmniejszenie zużycia paliwa i emisji.

\section{Przyszłość samochodów}

Motoryzacja stoi obecnie na progu nowej ery. Wraz z rozwojem sztucznej
inteligencji i komunikacji między pojazdami coraz bliższa staje się
wizja samochodów które będą mogły poruszać się bez
udziału kierowcy. Tego typu systemy analizują dane z kamer i radarów 
tworząc trójwymiarowy obraz otoczenia i podejmując decyzje w
czasie rzeczywistym.

Równocześnie rozwijane są co raz to nowecześniejsze systemy zarządzania ruchem oraz
infrastruktura drogowa która współpracująca z pojazdami. Dzięki temu przyszłe
samochody będą nie tylko bezpieczniejsze, ale również bardziej efektywne
energetycznie i przyjazne środowisku.


\section{Największe rekordy prędkości samochodó}

Poniższa tabelka przedstawia rekordy prędkości osiągnięte przez najszybsze auta produkcyjne na przełomie XX i XXI wieku:

\begin{table}[h]
\centering
\begin{tabular}{|l|c|c|c|c|}
\hline
Samochód                & Rok rekordu & Prędkość maks. & Moc silnika & Waga \\ \hline
Lamborghini Diablo      & 1990        & 326 km/h        & 492 KM      & 1385 kg \\ \hline
McLaren F1              & 1994        & 387 km/h        & 627 KM      & 1138 kg \\ \hline
Koenigsegg CCXR         & 2007        & 400 km/h        & 1018 KM     & 1180 kg \\ \hline
Hennessey Venom GT      & 2014        & 435 km/h        & 1244 KM     & 1244 kg \\ \hline
Koenigsegg Agera RS     & 2017        & 447 km/h        & 1341 KM     & 1295 kg \\ \hline
Bugatti Chiron SS       & 2019        & 490 km/h        & 1600 KM     & 1978 kg \\ \hline
\end{tabular}
\caption{Najszybsze samochody – porównanie}
\end{table}

\section{Podsumowanie}

Samochód to niezwykle złożony system, w którym współpracują mechanika,
elektronika i informatyka. Od prostych konstrukcji sprzed stu lat po
dzisiejsze hybrydy i auta elektryczne, przemysł motoryzacyjny przeszedł długą drogę
i ciągle się rozwija stając się symbolem postępu technologicznego. Zrozumienie  tego, 
jak działa samochód oraz jego podzespołów pozwala nam zrozumieć 
jak inżynierowie łączą naukę, technikę i innowacje, aby
tworzyć coraz to nowocześniejsze maszyny. Obecna autonomia i cyfryzacja 
powoduje zrównoważony rozwój, nacisk na obniżanie emisji spalin i rozwój
technologii. 

\chapter{Wnioski}

Wybrałem klasę Book ponieważ była dla mnie łatwiejsza do ustawienia, przez
większą hierarchię rozdziałów i podrodziałów, lepszą współpracę marginesów
oraz lepszy skład i oprawę.
Link do źródła: https://github.com/reazy02/Cwiczenie2.git

